\documentclass[conference]{IEEEtran}
\IEEEoverridecommandlockouts
% The preceding line is only needed to identify funding in the first footnote. If that is unneeded, please comment it out.
\usepackage{cite}
\usepackage{amsmath,amssymb,amsfonts}
\usepackage{algorithmic}
\usepackage{graphicx}
\usepackage{textcomp}
\usepackage{xcolor}
\def\BibTeX{{\rm B\kern-.05em{\sc i\kern-.025em b}\kern-.08em
    T\kern-.1667em\lower.7ex\hbox{E}\kern-.125emX}}
\begin{document}

\title{Smart Multi-Modal Soil and Crop Health Prediction System using Hybrid AI/ML Approach}

\author{\IEEEauthorblockN{1\textsuperscript{st} Given Name Surname}
\IEEEauthorblockA{\textit{Dept. of Computer Science} \\
\textit{SRM Institute of Science and Technology}\\
Chennai, India \\
email address or ORCID}
\and
\IEEEauthorblockN{2\textsuperscript{nd} Guide Name}
\IEEEauthorblockA{\textit{Dept. of Computer Science} \\
\textit{SRM Institute of Science and Technology}\\
Chennai, India \\
email address}
}

\maketitle

\begin{abstract}
Precision agriculture is critical for optimizing crop yields and conserving resources in the face of climate variability. This paper presents a "Smart Multi-Modal Soil and Crop Health Prediction System" that integrates Internet of Things (IoT) sensor data, real-time weather forecasts, and image analysis to provide actionable recommendations to farmers. Unlike traditional single-source models, our system employs a hybrid approach: combining Long Short-Term Memory (LSTM) networks for rainfall forecasting with a rule-based expert system for immediate soil intervention. The system architecture utilizes a microservices-based backend (FastAPI) and a responsive frontend (React) to deliver real-time insights on irrigation and fertilization. Preliminary results demonstrate the system's ability to successfully flag discrepancies between general weather forecasts and localized sensor data, providing a "Risk Alert" mechanism that enhances decision-making reliability.
\end{abstract}

\begin{IEEEkeywords}
Precision Agriculture, IoT, LSTM, Multi-modal Data, Soil Health, Machine Learning
\end{IEEEkeywords}

\section{Introduction}
Agriculture remains the backbone of many developing economies, yet it faces unprecedented challenges from unpredictable weather patterns and soil degradation. Traditional farming relies heavily on heuristic knowledge or generic weather reports, often leading to over-irrigation or improper fertilization...
[Expand on the need for Multi-modal data: Why just soil sensors aren't enough, why just weather API isn't enough.]

\section{Literature Survey}
[Placeholder: Summarize 2-3 papers on IoT in Agriculture and LSTM for weather prediction.]

\section{Proposed System Architecture}
The proposed system is designed as a cloud-native application comprising three core layers: Data Ingestion, Intelligence, and Presentation.

\subsection{Data Ingestion Layer}
We utilize a multi-modal approach:
\begin{itemize}
    \item \textbf{IoT Sensors:} NPK, pH, and Soil Moisture sensors transmit data via MQTT/HTTP.
    \item \textbf{Weather API:} Real-time integration with Open-Meteo for hyper-local forecasts.
    \item \textbf{Imagery:} Drone or smartphone captured images for visual health verification.
\end{itemize}

\subsection{Intelligence Layer}
The core logic resides in the Python-based backend.
\begin{equation}
    R_{final} = \alpha(M_{LSTM}) + \beta(E_{rule})
\end{equation}
Where the recommendation $R$ is a function of the ML prediction and the expert rule engine...

\section{Methodology}
\subsection{Data Collection}
Simulation of sensor nodes using randomized variability for robust testing...

\subsection{Hybrid Recommendation Engine}
The engine compares the API forecast with the local AI-driven history...

\section{Results and Discussion}
[Placeholder: Insert screenshot of the Dashboard here]
\begin{figure}[htbp]
\centerline{\fbox{Dashboard Screenshot Placeholder}}
\caption{User Dashboard showing Field Status and Recommendations.}
\label{fig}
\end{figure}

The system successfully identified water stress conditions 12 hours before visual wilting signs appeared in the test simulation...

\section{Conclusion}
The Smart Multi-Modal System successfully demonstrates the viability of combining deterministic rules with probabilistic ML models for safer agricultural advice...

\begin{thebibliography}{00}
\bibitem{b1} G. Eason, B. Noble, and I. N. Sneddon, ``On certain integrals of Lipschitz-Hankel type involving products of Bessel functions,'' Phil. Trans. Roy. Soc. London, vol. A247, pp. 529--551, April 1955.
\bibitem{b2} J. Clerk Maxwell, A Treatise on Electricity and Magnetism, 3rd ed., vol. 2. Oxford: Clarendon, 1892, pp.68--73.
\end{thebibliography}

\end{document}
